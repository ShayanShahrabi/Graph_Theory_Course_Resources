\documentclass{article}

\newtheorem{theorem}{Theorem}
\newtheorem{algorithm}{Algorithm}

\usepackage{hyperref}

% Language setting
\usepackage[english]{babel}

% Set page size and margins
\usepackage[letterpaper, top=2cm, bottom=2cm, left=3cm, right=3cm, marginparwidth=1.75cm]{geometry}

% Mathematical packages
\usepackage{amsmath}
\usepackage{amsfonts}
\usepackage{amssymb}

% Color package
\usepackage{xcolor}

\title{Session 2: Trees in Graphs}
\author{Shayan Shahrabi}
\date{November 2024}

\begin{document}
\maketitle

\section*{A Quick Review}

(In the following text, the symbol $T$ denotes an arbitrary tree.)

Definition of a tree: A connected graph without cycles!

\begin{theorem}
    In every graph resembling $T$, any two nodes, such as $x$ and $y$, are connected by \textbf{a unique} path.
    
    \textbf{Solution:} Prove by contradiction!
\end{theorem}

\begin{theorem}
    If $T$ has $p$ nodes and $q$ edges, it follows that \( q = p - 1 \).
    
    \textbf{Solution:} Use induction on $p$.
\end{theorem}

\begin{theorem}
    If $T$ is a binary tree with height $h$ and $p$ nodes, then the following holds:
    \[
    h + 1 \le p \le 2^{(h + 1)} - 1.
    \]

    \textbf{Solution:} Refer to page 103 of your textbook!
\end{theorem}

\begin{theorem}
    An m-tree (a tree in which each node has either $n$ children or is a leaf) with $l$ leaves has a height $h$ that satisfies the following inequality:
    \[
    h \ge \log_m(l).
    \]
\end{theorem}

\begin{algorithm}
    \textbf{Breadth First Search (BFS)}
\end{algorithm}

\begin{algorithm}
    \textbf{Depth First Search (DFS)} 
\end{algorithm}

\begin{algorithm}
    \textbf{Prim's Algorithm} \\
    For an explanation, view the following video: \href{https://www.youtube.com/watch?v=cplfcGZmX7I&t=2s}{Prim's Algorithm in 2 Minutes}.
\end{algorithm}

\begin{algorithm}
    \textbf{Kruskal's Algorithm} \\
    For an explanation, view the following video: \href{https://www.youtube.com/watch?v=71UQH7Pr9kU}{Kruskal's Algorithm in 2 Minutes}.
\end{algorithm}

\vspace{20pt}
\hrule
\vspace{20pt}

\section*{Problems}

\subsection*{Problem 1: Bye Bye Node!}

Suppose we remove a vertex $v$ and all edges connected to it from the tree $T$. Prove that if $\text{deg}(v) = w$, then $T$ is divided into $w$ connected components.

\subsection*{Problem 2: Rather Simple}

Prove that every connected graph contains a minimum spanning tree!

\subsection*{Problem 3: A Classic Problem}

Let $G$ be a graph with $p-1$ edges. Prove that the following statements are equivalent (TFAE):
\begin{itemize}
    \item $G$ is connected.
    \item $G$ has no cycles.
    \item $G$ is a tree.
\end{itemize}

\subsection*{Problem 4: No Name!}

Prove that a simple connected graph with exactly two intersecting edges forms a path.

\subsection*{Problem 5: One from Bondy!}

Prove that every tree with more than one node has at least two leaves.

\subsection*{Problem 6: Prove or Disprove}

Prove or disprove the following statement: If $d_1, d_2, \ldots, d_p$ are the degrees of vertices in a graph $G$, then the set of degrees \( \{1, d_1 + 1, d_2, \ldots, d_p\} \) is a valid set of degrees for a tree.

\subsection*{Problem 7: Tree of Cycles?!}

Suppose $T$ has $n$ vertices. Prove that $T$ is similar to a subgraph of $\bar{C}_{n+2}$.

\subsection*{Problem 8: Draw and Explore}

Draw a tree of your own choice and execute the BFS and DFS algorithms on it!

\subsection*{Problem 9: Let's Do Some Coding!}

You can take a look at this GitHub repository for further insights on the topic.

\end{document}
